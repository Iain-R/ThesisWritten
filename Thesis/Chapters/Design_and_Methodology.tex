%!TEX root = ../main.tex

\section{Modelling Data}
The modelling of data in this thesis is the vital first step to the optimization of a dragline, any simplifications made in this step will carry through all steps of the algorithm and effect the accuracy of the final results, therefore the manipulation of data to suit the system must be considered very carefully. As the dimensionality of the data decreases the runtime will as well, however the acheivable accuracy will decrease as the system will become more general. The method selected for the modelling of the geological data is a one dimensional density function used to represent the amount of overburden required to be removed from the mine along a specific strip. While this is not ideal in terms of a truly optimal solution it will allow the program to run rapidly, many assumptions along the way will result in loss of optimality at many stages, while an optimal solution is the desired result any feasible or near optimal solution is also acceptable. 
\\
Alternate methods that were considered for the modelling where a data driven approach or a two dimensional aproach, while a data driven aproach lends itself to machine learning of meta hueristics, these methods are often slow and not garunteed to lead to an optimal solution 
\\

One dimensional density functions are also useful as they can be generated through many methods, either through a pregenerated mine for testing or through geological survey data. In the thesis the sample data was generated by generating typical mine shapes and layouts of interest, such as a ramp and step function; within these samples a noise function was applied to give a better and more realistic set of data to test on. Survey data was supplied in a .xyz format, by assuming a constant depth of cut and a homogenius density of overburden the one dimensional density function was taken along the centre of the proposed strip path. This density data was represented as an Linked list for ease of use in python, an additional reason for this decisionis that the mine will always be mined in order and as such the list will simply be traveresed from start to end.  


\subsection{Types of Input}
The intended use of the program is to rapidly calculate optimal, near optimal or feasible block dimensions fora single strip inside a mine, however one of the key features of the program is its relaticly rapid runtimes, allowing on the fly adjustments and the ability to make corrections and compensations for errors. This input however is harder to predict and could be considered outside the scope of the project. As the thesis focuses more on the algorithms and underlying princples of the software rather than the user interface and user experience it is considered that the input to the project is a 1 dimensional array representing the spoil and another representing the mine at the current point in time, this will allow for expansion if required for on the fly adjustments. The program can also have the specifics of the dragline changed so that any possible dragline can be considered, while this is simple to change there is no user interface for the program and as such if it were to see further use then a graphc user interface would need to be constructed. \\

\subsection{Assumptions}

\section{Cost of Overburden Movement}
The cost of a block is vital to the solution to the problem, however the cost of a block is calculated as a summation of many individual movements, therefore the proper consideration for these models should be undertaken, ideally the cost of overburden movement will be able to calculate the time of movement for the action of moving overburden from point $n$ in the mine to $s$ in the spoil strip. Therefore two inputs would be the only required information for this function.
\subsection{Scope}
The function to calculate the cost of an action is required to work for all $n\in N , s \in S$ if this movement is at all valid, therefore we can assume that only valid enteries will be passed into this solution. Therefore for any valid combination the function should return a cost of movement, while the cost funtion is not required to be linear it must be able to be used in a Linear programming engine such as Gurboi, and as such must be Linearly related to block usage for objective funtions. This is the most pressing constraint as otherwise linear programming will be infeasible.
\subsection{Assumptions}
\subsection{Justification}

\section{Cost of a Block}
\subsection{Scope}
\subsection{Assumptions}
\subsection{Justification}

\section{Cost of a Strip}
\subsection{Scope}
\subsection{Assumptions}
\subsection{Justification}

\section{Communication of Results}
\subsection{Scope}
\subsection{Assumptions}
\subsection{Demonstration}