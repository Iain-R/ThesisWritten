%!TEX root = ../main.tex

The results of this thesis are intended to show evidence of a working system, and in particular the performance of subsystems. The validation and explanation of such subsystems is vital to the validity of the work done in this thesis. 
\section{Cost of Block}
In the implementation of this thesis three methods were selected to represent the cost of a block, from these three the Mixed integer linear program with validity constraints was selected to be the model used in the final iteration of the project. The model was then compared to the results generated by preexisting algoritms\cite{A*Search} such that a validation could be made, the comparison of these two methods is seen below in table \ref{tableres}. 

\begin{center}
% \caption{Validation of results}
\label{tableres}
 \begin{tabular}{|c| c| c|c|} 
 \hline
 Length of Block & Operation Cost& True Cost & Error (\%)\\ 
 32 & 7466 & 7377 & 1.206\\ 
 \hline
 37 & 8578 & 8383 & 2.326 \\
 \hline
 42 & 9743 & 9314 & 4.606 \\
 \hline
 47 & 10908 & 10229 & 6.638 \\
 \hline
 52 & 12020 & 11554 & 4.033\\ 
 \hline
\end{tabular}
\end{center}



As table \ref{tableres} shows the error between the true values and the calculated values increases as the length of the block increases, this could be in part due to differences and assumptions made in the modelling of the spoil chanel and spoil capacity. However it is also reasonable that this error is due in no small part to the assumptions made when reducing a three dimensional problem into two dimensions an as such could be seen as a potential validation of the model of block costing, furthermore the typical cost of a block for any enviroment seems to have similar results through all blocks, with the more complex datasets (5,6,7,8,9,10) having a larger cost in general than the simpler counterparts. As seen in table \ref{RESMIP} the results for sets 1,3,4,5 were all very similar, however the results for the more complex blocks are often larger as the range of feasible solutions is lower. This suggests that the cost of a block is dependant on location, legnth and spoil, and as such methods such as machine learning or linear simplifications may not be valid. 
\begin{center}
\begin{tabular}{ | c | c | c | c | c | c |}
\label{RESMIP}
% \caption{Compilation of results}
\hline
	Length & Set 1 & Set 2 & Set 3 & Set 4 & Set 5  \\ \hline
	32 & 7466.6 & 5772.1 & 7201.9 & 7784.4 & 7466.6  \\ \hline
	37 & 8578.7 & 6884.1 & 8261 & 9108.2 & 8472.8  \\ \hline
	42 & 9743.7 & 8419.8 & 9055.3 & 10696.9 & 9214.1 \\ \hline
	47 & 10908.7 & 10008.5 & 10114.4 & 12285.5 & 10008.5 \\ \hline
	52 & 12020.7 & 10855.7 & 11173.5 & 13768.3 & 11226.4 \\ \hline
	\hline
	Length & Set 6 & Set 7 & Set 8 & Set 9 & Set 10  \\ \hline
	32 & 7625.5 & 8261 & 7519.6 & 7360.7 & 9002.3  \\ \hline
	37& 8737.5 & 9320 & 8525.7& 8261 & 10326.2  \\ \hline
	42 & 10061.4 & 10485.1 & 9373 & 9214.1 & 11491.2 \\ \hline
	47 & 11173.5 & 12073.7 & 10167.3 & 9743.7& 12020.7 \\ \hline
	52 & 12020.7 & 13079.8 & 10961.6 & 11120.5 & 12762.1  \\ \hline
	\hline
	 % &  &  &  &  &  &  &  &  &  &  & \  & \  \\ \hline
\end{tabular}
\end{center}
However the cost of a block is not the only output generated by the model, the state of the spoil after the block has been removed is also returned. Spoil constraints on the overall solution require accurate and appropriate spoil functions to be generated 