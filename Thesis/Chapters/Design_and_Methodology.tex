%!TEX root = ../main.tex

\section{Modelling Data}
The modelling of data in this thesis is the vital first step to the optimization of a dragline, any simplifications made in this step will carry through all steps of the algorithm and effect the accuracy of the final results, therefore the manipulation of data to suit the system must be considered very carefully. As the dimensionality of the data decreases the runtime will as well, however the acheivable accuracy will decrease as the system will become more general. The method selected for the modelling of the geological data is a one dimensional density function used to represent the amount of overburden required to be removed from the mine along a specific strip. While this is not ideal in terms of a truly optimal solution it will allow the program to run rapidly, many assumptions along the way will result in loss of optimality at many stages, while an optimal solution is the desired result any feasible or near optimal solution is also acceptable. 
\\
Alternate methods that were considered for the modelling where a data driven approach or a two dimensional aproach, while a data driven aproach lends itself to machine learning of meta hueristics, these methods are often slow and not garunteed to lead to an optimal solution 
\\

One dimensional density functions are also useful as they can be generated through many methods, either through a pregenerated mine for testing or through geological survey data. In the thesis the sample data was generated by generating typical mine shapes and layouts of interest, such as a ramp and step function; within these samples a noise function was applied to give a better and more realistic set of data to test on. Survey data was supplied in a .xyz format, by assuming a constant depth of cut and a homogenius density of overburden the one dimensional density function was taken along the centre of the proposed strip path. This density data was represented as an Linked list for ease of use in python, an additional reason for this decisionis that the mine will always be mined in order and as such the list will simply be traveresed from start to end.  


\subsection{Types of Input}
The intended use of the program is to rapidly calculate optimal, near optimal or feasible block dimensions fora single strip inside a mine, however one of the key features of the program is its relaticly rapid runtimes, allowing on the fly adjustments and the ability to make corrections and compensations for errors. This input however is harder to predict and could be considered outside the scope of the project. As the thesis focuses more on the algorithms and underlying princples of the software rather than the user interface and user experience it is considered that the input to the project is a 1 dimensional array representing the spoil and another representing the mine at the current point in time, this will allow for expansion if required for on the fly adjustments. The program can also have the specifics of the dragline changed so that any possible dragline can be considered, while this is simple to change there is no user interface for the program and as such if it were to see further use then a graphc user interface would need to be constructed. \\


\section{Cost of Overburden Movement}
The cost of a block is vital to the solution to the problem, however the cost of a block is calculated as a summation of many individual movements, therefore the proper consideration for these models should be undertaken, ideally the cost of overburden movement will be able to calculate the time of movement for the action of moving overburden from point $n$ in the mine to $s$ in the spoil strip. Therefore two inputs would be the only required information for this function.
\subsection{Scope}
The function to calculate the cost of an action is required to work for all $n\in N , s \in S$ if this movement is at all valid, therefore we can assume that only valid enteries will be passed into this solution. Therefore for any valid combination the function should return a cost of movement, while the cost funtion is not required to be linear it must be able to be used in a Linear programming engine such as Gurboi, and as such must be Linearly related to block usage for objective funtions. This is the most pressing constraint as otherwise linear programming will be infeasible.
\subsection{Assumptions}
The assumptions made in this section is that the fucntion can calculate a reasonable model of the cost of movement through a continous function of one input. This will allow for the cost of a block to be calculated based on swingtime, however the movement of the dragline is not taken into account as this would be consistent for any action pattern and therefore can be negated. 
% \subsection{Justification}

\section{Cost of a Block}
The cost of a block is vital in the calculation of the cost of the entire strip, each block will have a cost dependant on its location within the mine, its size and the state of the spoil prior to exscavating the block. The function must be able to be calculated quickly as it will be called many times.
\subsection{Assumptions}
The assumptions are the at the cost of movement through a block can be ignored, and the swingtime will be representative of the cost of the block, furthermore that the cost of a block is able to be calculated in a reasonably accurate in a one dimensional enviroment. Further it is assumed that the block is mined sequentially with no consideration to the specific movement pattern of the mine, all overbudern will be removed from the current section before the next section is considered and the reach of the dragline will be considrered from teh location of this block.  
\subsection{Justification}
The modelling of a block in this simplistic method is valid as the most important attribute with the block model is that the results are analagous to other models, as long as the cost of a block vs length is similar to that of other models the strip calculation method is still applicable as it can be appliedwith any other block modelling method. However the model selected is a reasonable approximation of overburden removal for each block as is should represent the most basic of dragline movement patterns throughout the block. For more complex patterns this model must be updated, however typically more complex models will require a higher dimensionality of data and therefore this model would have to be reconsidered.  


\section{Cost of a Strip}
The cost of a strip must be calculated in a way such that the cost of all total blocks is considered and minimised, the cost of blocks as well as a constant for starting a new block is the only cost associated with the cost of a strip. This assumption means that the cost of a strip is independant on all other actions that preceed it but that the spoil is updated as the strip is mined, furthermore the length of the strip must be defined before the mining has begun such that the desired strip length is already defined as the allowable length of the mine. 
\subsection{Assumptions} 
The strip must be removed entirely and the length of the strip cannot be increased as this may be unallowable and cannot be decreased as the savings in time are offset by the loss of potential profit, furthermore it is assumed that there is some additional cost in time for the starting of a new block,associated with the setup of machinery and relocation of equipment. 
\subsection{Justification}
The suggested model is a valid application of ordering of blocks in a way to reduce the ost of actions independantly of the block model, this means that if the block model is changed the method of optimising the 

% \section{Communication of Results}
% \subsection{Scope}
% \subsection{Assumptions}
% \subsection{Demonstration}